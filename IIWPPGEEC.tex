%%%%%%%%%%%%%%%%%%%%%%%%%%%%%%%%%%%%%%%%%
% Template LaTeX
% Versão 0.1 (29/10/2018)
%
% Este template foi desenvolvido para o II WPPGEEC
% https://sites.google.com/view/upm-wppgeec/p%C3%A1gina-inicial
%
%%%%%%%%%%%%%%%%%%%%%%%%%%%%%%%%%%%%%%%%%

%---------------------------------------------------------------------------------
%%% Document configuration - Do not change
\documentclass[a4paper,12pt]{article}
\usepackage{geometry}
\geometry{
	a4paper,
	total={180mm,257mm},
	left=15mm,
	top=10mm,
}
\usepackage{fancyhdr}
\pagestyle{fancy}
\fancyhf{}
\fancyfoot[CE,CO]{WPPGEEC Proceedings, São Paulo, SP,  v.2, dez. 2018}
\fancyfoot[LE,RO]{\thepage}
\fancypagestyle{plain}{\pagestyle{fancy}}
\usepackage[utf8]{inputenc}
\usepackage{indentfirst}
%%%
%---------------------------------------------------------------------------------

%\usepackage[english]{babel} % Pacote para lingua inglesa
\usepackage[portuges]{babel} % Pcote para lingua portuguesa

\title{Paper Title (Título do Artigo)}
\author{%
	\textsc{Eduardo Santos}\\ % Primeiro autor
	\normalsize Universidade Presbiteriana Mackenzie \\ % Instituição primeiro autor
	\and
	\textsc{Gilberto Carvalho}\\% Segundo autor
	\normalsize Universidade de Sãp Paulo (USP) \\ % Instituição segundo autor
}
\date{\today}

\begin{document}
	\maketitle
	
	\begin{abstract}
		The abstract is limited to 50 words. This template provides authors with most of the formatting specifications needed for preparing electronic versions of their papers. Papers may be either in English or (Portuguese). All standard paper components have been specified, margins, column widths, line spacing, and type styles are built-in.
	
		\textbf{Keywords: Keyword I; Keyword II.} (Include up to 3 keywords here.)
	\end{abstract}
	
	\section{INTRODUCTION (INTRODUÇÃO)} \label{sec:introduction}
		The paper is limited up to 2 pages and should include an introduction with motivation and objectives, followed by the methods, partial results and conclusions. Examples of the type styles are provided throughout this document and are identified in italic type, within parentheses, following the example. Some components, such as multi-leveled equations, graphics, and tables are not prescribed, although the various table text styles are provided. The formatter will need to create these components, incorporating the applicable criteria that follow.
		
		%\textit{a)	Figures and Tables:} Place figures and tables at the top and bottom of columns. Avoid placing them in the middle of columns. Large figures and tables may span across both columns. Figure captions should be below the figures; table heads should appear above the tables. Insert figures and tables after they are cited in the text. Use the abbreviation Tab. \ref{tab:table1} for tables and Fig. \ref{fig:figure1} for figures, even at the beginning of a sentence.
		
		\begin{table}[!htb]
			\centering
			\begin{tabular}{|r|l|}
				\hline
				7C0 & hexadecimal \\
				3700 & octal \\ \cline{2-2}
				11111000000 & binary \\
				\hline \hline
				1984 & decimal \\
				\hline
			\end{tabular}
			\label{tab:table1}
			\caption{My caption}
		\end{table}
	
		%Figure Labels: Use 8 point Times New Roman for Figure labels. Use words rather than symbols or abbreviations when writing Figure axis labels to avoid confusing the reader. As an example, write the quantity “Magnetization”, or “Magnetization, M”, not just “M”. If including units in the label, present them within parentheses. Do not label axes only with units. In the example, write “Magnetization (A/m)” or “Magnetization {A[m(1)]}”, not just “A/m”. Do not label axes with a ratio of quantities and units. For example, write “Temperature (K)”, not “Temperature/K”.

	\section{METHODS (MÉTODOS)} \label{sec:method}
	
	\section{PARTIAL RESULTS (RESULTADOS PARCIAIS)} \label{sec:part_result}
	
	\section{CONCLUSIONS (CONCLUSÕES)}

	\begin{thebibliography}{9}
		\bibliographystyle{abbrv}
		\bibitem{Eason}
		G. Eason, B. Noble, and I. N. Sneddon,
		\newblock "On certain integrals of Lipschitz-Hankel type involving products of Bessel functions,” Phil. Trans. Roy. Soc. London, vol. A247, pp. 529–551, April 1955. (references)

		\bibitem{Clerk}
		J. Clerk Maxwell,
		\newblock A Treatise on Electricity and Magnetism, 3rd ed., vol. 2. Oxford: Clarendon, 1892, pp.68–73.
		
		
	\end{thebibliography}

\end{document}